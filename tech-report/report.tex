% !TEX TS-program = xelatexmk
% !TEX encoding = UTF-8 Unicode
% Compile this document using xelatex
\documentclass{report}

% Enable links in the PDF, but make them colored black
\usepackage[%
  colorlinks,
  linkcolor=black,
  anchorcolor=black,
  citecolor=black,
  filecolor=black,
  menucolor=black,
  runcolor=black,
  urlcolor=black,
  bookmarksopen=true,
  pdfpagelabels
]{hyperref}

% Make dates like this: 15 January 2013
\usepackage[english,cleanlook]{isodate}

% Make citations look like this: (Jones, 1979)
\usepackage{natbib}
\setcitestyle{round,semicolon}

% Make xelatex use latex mappings (such as --- for em-dash)
\usepackage{fontspec}
\defaultfontfeatures{Mapping=tex-text}


\title{Ducttape Technical Report}
\author{Lane Schwartz \and Jonathan Clark}

\begin{document}
\maketitle
\tableofcontents

\chapter{Motivation}

\citep{pedersen08}


\citep{deelmanetal09}

\citep{schwartz10} using GNU Make \citep{gnumake}

LoonyBin \citep{clarklavie10,clarketal10}

EMS \citep{koehn10ems}
  
Ducttape \citep{ducttape}

\chapter{The ducttape hyperworkflow language}

\chapter{HyperDAG theory}

\section{DAG --- directed acyclic graphs}

\section{HyperDAG --- directed acyclic hypergraphs}

\section{MetaHyperDAG --- directed acyclic hypergraphs with epsilon vertices}

\section{PhantomMetaHyperDAG --- directed acyclic hypergraphs with epsilon and phantom vertices }



\chapter{Building a workflow}

\section{Abstract syntax trees}

\begin{enumerate}
\item Parse the ducttape text file(s) into an abstract syntax tree (AST)
\end{enumerate} 

\section{Branch points and branches}

\begin{enumerate}

\item Traverse the AST to identify all branch points used in the workflow 
      (see \texttt{findBranchPoints()} in \texttt{WorkflowBuilder.build()}; also 
       see \texttt{BranchPoint} and \texttt{BranchPointFactory})

\item For each branch point, identify all branch names associated with that branch point 
      (see \texttt{Branch} and \texttt{BranchFactory}, especially \texttt{BranchFactory.getAll()})

\end{enumerate}


\section{Building task templates}

\subsection{Found tasks}

See \texttt{TaskTemplateBuilder.findTasks()}, called from \texttt{WorkflowBuilder.build()}

\begin{enumerate}

\item Gather a list of all task definitions

\item By iterating over this list, construct a map where 
      the keys are globally unique task names and 
      the values are task definitions

\item For each task, iterate over all input specifications and parameter specifications for that task
      (see \texttt{TaskTemplateBuilder.findTasks()})

\item This step is extremely complicated and involves several mutually recursive functions.
      For each input specification and parameter specification in a task, 
      collect the set of possible values that each input and parameter can take. 
      This set of possible values will be recorded as a \texttt{BranchPointTree}.
      See \texttt{TaskTemplateBuilder.resolveBranchPoint()}, called from \texttt{TaskTemplateBuilder.findTasks()}.
      
\item BranchGraft elements are currently handled within \texttt{TaskTemplateBuilder.resolveNonBranchVar()}

\end{enumerate}


\section{Temporal and structural dependencies}

\section{Constructing a packed hyperDAG}

See \texttt{WorkflowBuilder.build()}

\begin{enumerate}
\item For each task template, create a vertex. Store these vertices in a map, where the keys are globally-unique task names.
\end{enumerate}


\bibliographystyle{plainnat}
\bibliography{report}

\end{document}